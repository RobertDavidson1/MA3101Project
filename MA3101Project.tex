\documentclass[11pt]{article}
\usepackage{changepage}
\usepackage{hyperref}
\usepackage{amssymb}
\usepackage{amsmath}
\begin{document}

\begin{center}
    \vspace{0.5cm}
    Investigating the Acceleration Vector and Geodesics on $S^2$\\
    MA3101 Project\\[24pt]
    \LARGE

    \Large
    \textbf{Robert Davidson}\\[6pt]
    \small
    BSc Mathematics and Computer Science \\ r.davidson1@universityofgalway.ie\\[6pt]
    Date: \today\\[12pt]
\end{center}

\vspace{1.5cm}

\subsection*{Why this project?}
In my Euclidean and Non Euclidean Geometry lectures, we learned that if the acceleration vector of a curve is always normal to the tangent plane, then the curve is a geodesic\\[0.5ex]
On the sphere, $S^2$, every geodesic is a great circle, which is circle formed a plane through the sphere's center. \\[0.5ex]
I was really intrigued by this result, and I wanted to find a way to visualise it, which led me to build an interactive visualization, which is avilable at \href{https://robertdavidson1.github.io/MA3101Project/}{this website} hosted on GitHub.
This brief writeup covers the mathematics I did to both understand why the statement is true, and to create the website.

\subsection*{The Motivation}
In order to explore the statement, that all geodesics on $S^2$ have acceleration vectors normal to the tangent plane, we first need to define the geometry of $S^2$ and what it means for a vector to be normal to the tangent plane. \\[0.5ex] We then move onto to defining the plane so we can define how it intersects the sphere. \\[0.5ex]
Once we have this plane, we can define a circle on this plane, and parameterize it, and take the derivatives in order to get the velocity and acceleration vectors. \\[0.5ex]
Finally, we tie it all together to show that if the acceleration vector of a curve is always normal to the tangent plane, then the curve is a geodesic.

\newpage

\subsection*{The 2-Sphere}
We define the 2-sphere, $S^2$, as the set of all points in $\mathbb{R}^3$ such that:
$$x^2 + y^2 + z^2 = 1$$
This represents a sphere of radius 1 centered at the origin. \\
The normal vector at a point $\mathbf{s} = (x, y, z)$ on the 2-sphere is the vector perpendicular to the tangent plane at that point. \\
It is given by the gradient of the function $f(x,y,z) = x^2 + y^2 + z^2 - 1$:
$$\mathbf{n}_{S^2} = \nabla f(\mathbf{s}) = (2x, 2y, 2z) = 2\textbf{s}$$
So, the normal vector is just a scalar multiple of the position vector.
\subsection*{The Plane}
A plane $P$ can be written as:
$$ax + by + cz = d$$
Where $\textbf{n} = (a,b,c)$ is the normal vector to the plane, so the equation becomes:
$$P : \textbf{n} \cdot (x,y,z) = d$$
We normalize $\textbf{n}$ so it represents direction only:
$$\hat{\textbf{n}} = \frac{\textbf{n}}{\|\textbf{n}\|}$$
We can then write the plane equation as:
$$P : \hat{\textbf{n}} \cdot (x,y,z) = \frac{d}{\|\textbf{n}\|}$$
We let $h = \frac{d}{\|\textbf{n}\|}$ be the distance from the origin to the plane.
So the plane equation becomes:
$$P : \hat{\textbf{n}} \cdot (x,y,z) = h$$
We note that the plane intersects the sphere when $|h| \leq 1$.
\newpage
\subsection*{The Intersection of a Plane and the 2-Sphere}
We define a line $L(t)$ a line from the origin in the direction of the planes normal vector: $L(t) = t \hat{\textbf{n}}$. \\
The line intersects the plane when $t= h$, so the closest point on the plane to the origin is:
$$\mathbf{C} = h \hat{\mathbf{n}}
$$
The plane cuts the sphere in a circle centered at $\mathbf{C}$.
If we let:
\noindent



\vspace{0pt}
\begin{itemize}
    \item Q : A point on the circumference of the circle formed by the intersection of the plane and the sphere
    \item C : The center of the circle
    \item O : The origin
\end{itemize}
We form a $\triangle OQC$ with sides:
\begin{itemize}
    \item $OC = h$ : Distance from origin to center of circle
    \item $OP = R$ : Distance from origin to point on circumference
    \item $CP = r$ : Distance from center of circle to point on circumference of planes cut
\end{itemize}
By the Pythagorean theorem, we have:
\begin{align*}
    R^2 & = h^2 + r^2                                               \\
    r   & = \sqrt{R^2 - h^2}                                        \\
    r   & = \sqrt{1 - h^2} \quad \text{Since radius of sphere is 1}
\end{align*}
So the circle on the sphere, cut by plane has center, $\mathbf{C} = (0,0,h)$ and radius $r = \sqrt{1 - h^2}$. \\
For simplicty, in the website, we let $\hat{\textbf{n}} = (0,0,1)$, which also helped in choosing vectors $\mathbf{u}$ and $\mathbf{v}$ later, then we get
\begin{align*}
     & P : (0,0,1) \cdot (x,y,z) = h             \\
     & P : z = h                                 \\[1ex]
     & \mathbf{C} = h \hat{\textbf{n}} = (0,0,h) \\
     & r = \sqrt{1 - h^2}
\end{align*}
\subsection*{Parameterization of the Circle on the Plane}
We choose two vectors, \textbf{u} and \textbf{v}, that are on the plane, that is:
$$\mathbf{u}, \textbf{v} \perp \hat{\textbf{n}}$$
If we enforce $$\mid \mid \mathbf{u} \mid \mid = \mid \mid \textbf{v} \mid \mid = 1 \quad \text{and} \quad \mathbf{u} \perp \textbf{v}$$
we can define a circle on P:
$$\textbf{p}(t) = \cos(t) \mathbf{u} + \sin(t) \textbf{v} \quad t \in [0, 2\pi]$$
More precisely, $\textbf{p}(t)$ is a position vector from the origin to a point on the circle on the plane. \\
Recognising that we let $\hat{\textbf{n}} = \hat{k} = (0,0,1$), and using the fact:
$$\hat{i} \perp \hat{j} \perp \hat{k} \quad \text{and} \quad \mid \mid \hat{i} \mid \mid = \mid \mid \hat{j} \mid \mid = \mid \mid \hat{k} \mid \mid = 1$$
We can let $\textbf{u} = \hat{i}$ and $\textbf{v} = \hat{j}$, then we get:
$$\textbf{p}(t) = \cos(t) \hat{i} + \sin(t) \hat{j}\quad t \in [0, 2\pi]$$
Finally, scaling to radius $r$ and moving to center $\mathbf{c}$ we get:
$$\textbf{p}(t) = \mathbf{C} + r \left[\cos(t) \hat{i} + \sin(t) \hat{j}\right]$$
\subsection*{Velocity and Acceleration Vectors}
We can take the first and second derivatives of $\textbf{p}(t)$ to get the velocity and acceleration vectors:
\begin{align*}
    \mathbf{v}(t) & = \frac{d}{dt} \left[\textbf{p}(t)\right] =  r\left[-\sin(t) \hat{i} + \cos(t) \hat{j} \right] \\
    \mathbf{a}(t) & = -r[\cos(t)\hat{i} + \sin(t)\hat{j}]
\end{align*}
Rewriting \textbf{p}(t), we get:
$$r[\cos(t)\,\hat{i} + \sin(t)\,\hat{j}] = \textbf{p}(t) - \mathbf{C}$$
So, we can rewrite $\mathbf{a}(t)$ as:
$$\mathbf{a}(t) = \mathbf{C} - \textbf{\textbf{p}(t)}$$
\newpage
\subsection*{Tieing it all together}
We see that if $h = 0$, that is the distance from the origin to the plane is 0, the plane gives us a circle on $S^2$ with center at the origin and radius 1, which is the definition of a great circle.
\begin{align*}
    \mathbf{C} & = h \hat{\textbf{n}} = 0 \hat{\textbf{n}} = 0 \\[1ex]
    r          & = \sqrt{1- h^2} = \sqrt{1- 0^2} = 1           \\
\end{align*}
The aceleration vector also simplifies to:
\begin{align*}
    \mathbf{a}(t) & = \mathbf{C} - \textbf{p}(t) \\
    \mathbf{a}(t) & = 0 - \textbf{p}(t)          \\
    \mathbf{a}(t) & = - \textbf{p}(t)
\end{align*}
So when $h = 0$,that is, when the plane cuts the 2-Sphere through the origin, we get a circle centered at the origin with radius 1, this by definition is a great circle. \\
The acceleration vector around this circle is just a scalar multiple of the position vector, which is the same as the sphere's normal vector. \\
Therefore, in order for the acceleration vector to be normal to the tangent plane, the curve must be a great circle, and thus a geodesic.
\pagebreak
\section*{Acknowledgements}
I used ChatGPT to help verify the algebraic derivations and ensure the mathematical reasoning in this write-up was correct.

\end{document}